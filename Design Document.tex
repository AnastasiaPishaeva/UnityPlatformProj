\documentclass{article}
\usepackage[fontsize=13pt]{fontsize}
\usepackage{geometry}
\usepackage{graphicx} % Required for inserting images
\usepackage[T2A]{fontenc}
\newgeometry{top=20mm, bottom=20mm, left=10mm, right=20mm}
\title{Дизайн документ}
\author{Пищаева Анастасия \\ Игонина Ксения \\ Иконников Андрей \\ Хохлов Тимофей \\ Аветисян Гаспар}
\date{Ноябрь 2024}

\begin{document}

\maketitle

\tableofcontents
\newpage

\section{Введение}
Этот документ содержит описание концепции и дизайна игры "Гномик в лесу", предназначенной для мобильных устройств. В нем описаны ключевые аспекты игры, включая геймплей, механики, мир, персонажей, а также технические детали разработки. Данный документ служит основой для дальнейшего планирования и реализации проекта.

\begin{itemize}
\item Комментарии по организации содержимого документа

Документ разделен на несколько основных разделов, каждый из которых описывает определенные аспекты игры:
\begin{itemize}

\item Введение - краткая формулировка всей идеи игры

\item Жанр и аудитория — сведения о целевой аудитории и жанре

\item Основные особенности игры — ключевые особенности и примерный объем игры

\item Описание игры — основная концепция

\item Предпосылки создания — "право на жизнь" для игры

\item Платформа — на чем планируется запуск игры
\end{itemize}

Дизайн-документ ориентирован на команду разработчиков и дизайнеров, а также на всех, кто будет участвовать в процессе создания игры.

\item Ссылки на использованные материалы, копирайты и прочее
\begin{itemize}


\item Концепт-арт и иконки: Все изображения, использованные в игре, были созданы дизайнером нашей команды.

\item Звуковые эффекты и музыка: Для разработки использована нелицензированная музыка.

\item Графика: Все использованные графические элементы являются оригинальными.
\end{itemize}

\item История изменений документа
\begin{center}
\begin{tabular}{| c | c | c | c |}
\hline
 Версия & Дата & Автор & Изменение \\  \hline
 1.0 & 2024-11-17 & Анастасия Пищаева & Написано 4 пункта введения для документа\\ \hline
 1.1 & 2024-11-17 & Игонина Ксения & Дописано введение и Концепт и Введение\\  \hline 
 1.2 & 2024-11-17 & Аветисян Гаспар & Написаны Жанр и аудитория и Основные особенности игры\\  \hline
  & & & \\  \hline
  & & & \\  \hline 
\end{tabular}
\end{center}

\item Список авторов
	\begin{itemize}
		\item Андрей Иконников — Главный разработчик.
		\item Тимофей Хохлов — Ответственный за программирование и техническую сторону проекта.
		\item Ксения Игонина — Ответственный за программирование и техническую сторону проекта.
		\item Анастасия Пищаева — Дизайнер.
		\item Гаспар Аветисян — Создатель музыкального сопровождения.
	\end{itemize}

\item Условные обозначения, сокращения и соглашения
	\begin{itemize}
            \item Гномик — Главный персонаж игры.
            \item Раннер — Тип игры, в которой игрок управляет персонажем, который автоматически движется вперед.
            \item UI — Пользовательский интерфейс.
            \item UX — Опыты пользователя.
	\end{itemize}

\item Другие сведения, необходимые для прочтения документа
	\begin{itemize}
            \item Для лучшего понимания дизайна игры рекомендуется ознакомиться с аналогичными играми жанра «раннер» на мобильных платформах, чтобы учитывать лучшие практики и выявить потенциальные улучшения.
            \item Раннер — Тип игры, в которой игрок управляет персонажем, который автоматически движется вперед.
            Дополнительные материалы, такие как концепт-арты и описание прототипа, могут быть предоставлены отдельно по запросу.
	\end{itemize}
\end{itemize}

\section{Концепция}

Этот документ описывает ключевые аспекты игры "Гномик в лесу", которая будет разработана как мобильный аркадный раннер с элементами платформера. Концепт-файл представляет собой обзор основных механик игры, целевой аудитории, особенностей и обоснования для её создания.

\subsection{Введение}

"Гномик в лесу" — это мобильная аркадная игра в жанре раннера, где игрок управляет маленьким гномом, который бежит через лес, избегая препятствий и собирая бонусы. Игра не имеет уровней, а её цель — пройти как можно дальше, увеличивая свой рекорд с каждым новым запуском. Лес всегда одинаков, но сложность игры возрастает с течением времени, предлагая игроку всё более сложные препятствия и вызовы. Игра ориентирована на детей и взрослых, обеспечивая простое управление и бесконечный игровой процесс.

\subsection{Жанр и аудитория}

\begin{itemize}
    \item \textbf{Жанр}: платформер с элементами приключения.
    \item \textbf{Возрастные ограничения}:
    \begin{itemize}
        \item \textit{7+}: сдержанный стиль с элементами фэнтези, не содержащий жестоких сцен.
    \end{itemize}
\end{itemize}

\subsection{Основные особенности игры}

\subsubsection{Ключевые особенности (USP)}
\begin{enumerate}
    \item \textbf{Интуитивно понятный геймплей:} простой и интуитивно понятный геймплей: Игра подходит для игроков всех возрастов благодаря простоте управления. Гномик автоматически бежит, а игроку нужно лишь уклоняться от препятствий и собирать бонусы с помощью свайпов или тапов.
    \item \textbf{Однородный фон:} в отличие от множества игр с меняющимися локациями, наш лес остаётся неизменным на протяжении всей игры. Это позволяет сосредоточиться на чистом и бесперебойном игровом процессе, где главной задачей становится преодоление все более сложных препятствий.
    \item \textbf{Бесконечный режим:} игра не имеет четкой финальной цели, и ее суть заключается в том, чтобы пройти как можно дальше. С каждым новым рекордом игрок сталкивается с более сложными препятствиями, но сам мир и фон остаются одинаковыми, что подчеркивает чисто аркадный и бесконечный характер игры.
    \item \textbf{Не требуется улучшение героя:} в отличие от многих игр, где игрок должен развивать персонажа и открывать новые способности, в нашей игре гномик остается неизменным. Это упрощает игру и делает её более доступной для быстрого старта и непрерывных сессий.
    \item \textbf{Фокус на рефлексах и внимании:} из-за того, что изменения в игровом процессе ограничены только увеличением сложности препятствий, игроку нужно опираться исключительно на свои рефлексы и концентрацию, что делает каждый момент на пути уникальным.
    \item \textbf{Уникальный саундтрек и звуковые эффекты:} Каждый игровой момент сопровождается атмосферной музыкой и звуками, создающими ощущение пребывания в волшебном лесу. Это добавляет игре особую атмосферу, несмотря на однообразие фона.
    \item \textbf{Доступность и легкость:} Игра не требует долгих сеансов и сложных решений. Она идеально подходит для коротких игровых сессий, что делает её отличным выбором для игры на ходу.
\end{enumerate}

\subsubsection{Примерный объем игры}
\begin{itemize}
    \item \textbf{Время прохождения:} из-за бесконечного игрового процесса с возможностью достижения рекордов, игра предоставляет игрокам практически неограниченное количество сессий. Среднее время одного игрового сеанса — 5-10 минут, но в зависимости от умения игрока, каждый новый запуск может длиться дольше.
\end{itemize}




\subsection{Описание игры}

\subsection{Предпосылки создания}

\subsection{Платформа}

\section{Функциональная спецификация}

\subsection{Принципы игры}

\subsubsection{Суть игрового процесса}

\subsubsection{Ход игры и сюжет}

\subsection{Физическая модель}

\subsection{Персонаж игрока}

\subsection{Элементы игры}

\subsection{«Искусственный интеллект»}

\subsection{Многопользовательский режим}

\subsection{Интерфейс пользователя}

\subsubsection{Блок-схема}

\subsubsection{Функциональное описание и управление}

\subsubsection{Объекты интерфейса пользователя}

\subsection{Графика и видео}

\subsubsection{Общее описание}

\subsubsection{Двумерная графика и анимация}

\subsubsection{Трехмерная графика и анимация}

\subsubsection{Анимационные вставки}

\subsection{Звуки и музыка}

\subsubsection{Общее описание}

\subsubsection{Звук и звуковые эффекты}

\subsubsection{Музыка}

\subsection{Описание уровней}

\subsubsection{Общее описание дизайна уровней}

\subsubsection{Диаграмма взаимного расположения уровней}

\subsubsection{График введения новых объектов}

\section{Контакты}


\end{document}
