\documentclass{article}
\usepackage[fontsize=13pt]{fontsize}
\usepackage{geometry}
\usepackage{graphicx} % Required for inserting images
\usepackage[T2A]{fontenc}
\newgeometry{top=20mm, bottom=20mm, left=10mm, right=20mm}
\title{Дизайн документ}
\author{Пищаева Анастасия \\ Игонина Ксения \\ Иконников Андрей \\ Хохлов Тимофей \\ Аветисян Гаспар}
\date{Ноябрь 2024}

\begin{document}

\maketitle

\tableofcontents
\newpage

\section{Введение}
Этот документ содержит описание концепции и дизайна игры "Гномик в лесу", предназначенной для мобильных устройств. В нем описаны ключевые аспекты игры, включая геймплей, механики, мир, персонажей, а также технические детали разработки. Данный документ служит основой для дальнейшего планирования и реализации проекта.

\begin{itemize}
\item Комментарии по организации содержимого документа

Документ разделен на несколько основных разделов, каждый из которых описывает определенные аспекты игры:
\begin{itemize}

\item Введение - краткая формулировка всей идеи игры

\item Жанр и аудитория — сведения о целевой аудитории и жанре

\item Основные особенности игры — ключевые особенности и примерный объем игры

\item Описание игры — основная концепция

\item Предпосылки создания — "право на жизнь" для игры

\item Платформа — на чем планируется запуск игры
\end{itemize}

Дизайн-документ ориентирован на команду разработчиков и дизайнеров, а также на всех, кто будет участвовать в процессе создания игры.

\item Ссылки на использованные материалы, копирайты и прочее
\begin{itemize}


\item Концепт-арт и иконки: Все изображения, использованные в игре, были созданы дизайнером нашей команды.

\item Звуковые эффекты и музыка: Для разработки использована нелицензированная музыка.

\item Графика: Все использованные графические элементы являются оригинальными.
\end{itemize}

\item История изменений документа
\begin{center}
\begin{tabular}{| c | c | c | c |}
\hline
 Версия & Дата & Автор & Изменение \\  \hline
 1.0 & 2024-11-17 & Анастасия Пищаева & Написано 4 пункта введения для документа\\ \hline
  & & & \\  \hline 
  & & & \\  \hline
  & & & \\  \hline
  & & & \\  \hline 
\end{tabular}
\end{center}

\item Список авторов
	\begin{itemize}
		\item Андрей Иконников — Главный разработчик.
		\item Тимофей Хохлов — Ответственный за программирование и техническую сторону проекта.
		\item Ксения Игонина — Ответственный за программирование и техническую сторону проекта.
		\item Анастасия Пищаева — Дизайнер.
		\item Гаспар Аветисян — Создатель музыкального сопровождения.
	\end{itemize}
\end{itemize}


\section{Концепция}

\subsection{Введение}

\subsection{Жанр и аудитория}

\subsection{Основные особенности игры}

\subsection{Описание игры}

\subsection{Предпосылки создания}

\subsection{Платформа}

\section{Функциональная спецификация}

\subsection{Принципы игры}

\subsubsection{Суть игрового процесса}

\subsubsection{Ход игры и сюжет}

\subsection{Физическая модель}

\subsection{Персонаж игрока}

\subsection{Элементы игры}

\subsection{«Искусственный интеллект»}

\subsection{Многопользовательский режим}

\subsection{Интерфейс пользователя}

\subsubsection{Блок-схема}

\subsubsection{Функциональное описание и управление}

\subsubsection{Объекты интерфейса пользователя}

\subsection{Графика и видео}

\subsubsection{Общее описание}

\subsubsection{Двумерная графика и анимация}

\subsubsection{Трехмерная графика и анимация}

\subsubsection{Анимационные вставки}

\subsection{Звуки и музыка}

\subsubsection{Общее описание}

\subsubsection{Звук и звуковые эффекты}

\subsubsection{Музыка}

\subsection{Описание уровней}

\subsubsection{Общее описание дизайна уровней}

\subsubsection{Диаграмма взаимного расположения уровней}

\subsubsection{График введения новых объектов}

\section{Контакты}


\end{document}